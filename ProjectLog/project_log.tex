\documentclass[]{article}

\usepackage{titlesec}
\usepackage[margin=1in]{geometry}

\newenvironment{logentry}[2]% date, heading
{\noindent\vspace{.2cm}\large\textbf{#2}\titlerule\large\textbf{#1}}
{\\\\\\}

\begin{document}
\begin{logentry}{01/20/20}{{\LaTeX}, Google Format, and DE}
    \par The first week after officially starting my project has been busy.
    I started with trying to improve my development environment in the 
    hopes that it will make writing the project easier.  I'm using vim
    for all of my development.  Vim has a lot of nice plugins that I've 
    added to make my life easier.  For example, Ultisnips which allows
    for the creation and insertion of code snippets to streamline
    coding and prevent you from writing the same thing over and
    over again.
    \par Once I was mostly happy with my setup, I turned my attention
    to my project.  I've written a good chunk of code already for it
    from when I was in CS616, but it needed some work.  I spent some
    time cleaning up what was already there and making the project more
    concise.  For example, I was caching the important words from each of 
    the scanned source files in my home directory, so I moved it to the root
    of the project structure instead.  I also moved my git out a level so that
    I could include the testing directory - a future project, along with other
    fixes.
    \par The biggest fix to my code was formatting.  I had my own internalized 
    style when writing, so it was fairly consistent, but it wasn't perfect.
    While figuring out my DE I added a plugin for linting and got it working
    for the project's language - Java.  I went through and updated all of my
    code to follow Google's standard for Java development, minus their rule
    on tabs.  The indentation is correct, but I'm a fan of tabs over spaces 
    as all sane humans should be.  All that's left for that is writing the
    Javadoc comments.  Once that's complete it'll all be compliant.
    \par The last thing I worked on was learning {\LaTeX}. In all my
    time at university I've never sat down to learn {\LaTeX}, but with
    such a long term writing project - these logs - I decided
    to invest the time and I can already tell it was a good choice.
    {\LaTeX} is very powerful and I've got it automated with plugins 
    Ultisnips and vimtex, so that it is extremely easy to update when I make 
    changes to the project.
\end{logentry}
\begin{logentry}{02/01/20}{Cutting Fat and Cleaning Code}
    \par The main focus for my recent work has been cleaning up the 
    code and removing unnecessary pieces.  I've removed some code that was
    responsible for keeping track of the top scorers.  Now, everything is map
    based.  I sort the map and pass a sub-map back with the top results.
    No more custom code solutions, no need to reinvent the wheel.
    \par I've also been working on documentation.  I finished
    all of the Javadoc comments for my code and I've done
    the last round of formatting.
    \par I've shifted around where certain operations are done.
    For example, I was doing file processing in file tracking for important
    word generation.  I would pass chunks of the file to the
    important words generator and it would hand back a map.
    I'd feed that map and another chunk back into the generator.
    This added a lot of unnecessary code to the file tracking.
    I've moved this functionality to the important words generator;
    now, I can feed it a file and it returns the map for the entire file.
    \par The last major change I've done is update the project to use
    a newer version of Java.  Newer Java versions added some functionality
    that I wanted to use for this project, like try-with-resources.
    \par The next step is to implement Unit Testing.  The code is in a 
    decent spot right now and it's important to write tests before
    I expand it any further.
\end{logentry}
\begin{logentry}{02/17/20}{Bug Fixing and Testing}
    \par When I started trying to implement JUnit tests in maven for the project
    I ran into a wall.  There were a lot of problems with my pom file and
    dependencies.  I'm not sure if it was the transition to the newer version
    of Java that broke certain pieces or that some of the dependencies
    that the Jenkins pom shipped with were out of date, probably
    a combination.  
    \par The problems with maven are now fixed.  JUnit is working and
    in the process I found some problems with my code.  When I fixed
    the pom a module that should've been working since the start 
    began to run, the bug detection module.  It scans the code for 
    potential problems or bugs.  When I fixed the pom I was suddenly 
    alerted to several problem areas in my code that needed fixing
    before I continued working on testing.
    \par Test development is now the main focus.  I've written
    most of the testing for the ImportantWordsGenerator at this point.
    Once that is all done, I'm going to work on getting a measure of 
    project efficiency.  That is, how quickly my software is able
    to process a project and return the top matching files.
    I'll be using Defects4j to supply the source code to test.
\end{logentry}
\end{document}
